\documentclass[12pt,a4paper]{report}

% Encoding and language
\usepackage[utf8]{inputenc}
\usepackage[T1]{fontenc}
\usepackage[english]{babel}

% Layout
\usepackage{geometry}
\geometry{margin=1in}
\usepackage{setspace}
\onehalfspacing

% Graphics, tables, code
\usepackage{graphicx}
\usepackage{booktabs}
\usepackage{longtable}
\usepackage{listings}
\usepackage{xcolor}

% Hyperlinks
\usepackage[colorlinks=true,linkcolor=blue,citecolor=blue,urlcolor=blue]{hyperref}

% Bibliography (choose your backend; biblatex here)
\usepackage[backend=biber,style=ieee]{biblatex}
\addbibresource{references.bib}

% Code block style
\lstset{
  basicstyle=\ttfamily\small,
  breaklines=true,
  frame=single,
  numbers=left,
  numberstyle=\tiny,
  keywordstyle=\color{blue},
  commentstyle=\color{gray},
  stringstyle=\color{teal}
}

% Title data (fill these)
\newcommand{\ProjectTitle}{<Your Project Title>}
\newcommand{\AuthorName}{<Your Name>}
\newcommand{\StudentID}{<Your Student ID>}
\newcommand{\StudyProgramme}{Biomedical Engineering}
\newcommand{\Specialization}{Information Systems in Medicine}
\newcommand{\SupervisorName}{Dr inz. Marcin Rudzki}
\newcommand{\SupervisorDept}{Department of Medical Informatics and Artificial Intelligence}
\newcommand{\SupervisorFaculty}{Faculty of Biomedical Engineering}
\newcommand{\ThesisYear}{2025}
\newcommand{\LogoPath}{logo.png} % place the provided logo at thesis/logo.png

\begin{document}

\begin{titlepage}
  \centering
  \vspace*{0.5cm}
  \IfFileExists{\LogoPath}{\includegraphics[width=0.35\textwidth]{\LogoPath}\par}{}
  \vspace*{1cm}
  {\Large BACHELOR PROJECT\par}
  \vspace{1.5cm}
  {\LARGE ``\ProjectTitle''\par}
  \vspace{2cm}
  {\Large \AuthorName\par}
  {\large \StudentID\par}
  \vspace{1cm}
  {\large Study programme: \StudyProgramme\par}
  {\large Specialization: \Specialization\par}
  \vspace{2cm}
  {\large SUPERVISOR\par}
  {\large \SupervisorName\par}
  {\large \SupervisorDept\par}
  {\large \SupervisorFaculty\par}
  \vfill
  {\Large ZABRZE \ThesisYear\par}
\end{titlepage}

\pagenumbering{roman}
\tableofcontents
\clearpage

\begin{abstract}
Summarize the motivation, method, key results, and contributions in 150--250 words.
\end{abstract}

\clearpage
\pagenumbering{arabic}

\chapter{Introduction}
\begin{itemize}
  \item Problem statement and context (rehabilitation, remote monitoring, wearables).
  \item Objectives and research questions/hypotheses.
  \item Contributions (e.g., Fitbit-integrated rehab app, pose estimation, patient dashboard).
  \item Thesis structure overview.
\end{itemize}

\chapter{Background and Related Work}
\begin{itemize}
  \item Digital health / tele-rehabilitation platforms and adherence.
  \item Wearables and Fitbit API (OAuth2, scope, data availability/rate limits).
  \item Pose estimation for exercise monitoring (e.g., MoveNet).
  \item Brief survey of similar systems and gaps.
\end{itemize}

\chapter{System Design}
\begin{itemize}
  \item Architecture overview (frontend React, backend Express, MongoDB).
  \item Data flow: authentication, Fitbit OAuth2, HR polling, result submission.
  \item Sequence diagrams for dashboard HR polling and exercise submission.
  \item Security/privacy considerations (tokens, storage, roles).
\end{itemize}

\chapter{Implementation}
\begin{itemize}
  \item Frontend: PatientDashboard, ExerciseRunner (pose detection, video capture, HR display).
  \item Backend: Fitbit routes (PKCE flow, summary/intraday fallbacks, caching), results API.
  \item Data models: users, exercises, results, notifications (include schema snippets).
  \item Error handling and fallbacks (rate limits, cached HR, summary HR).
\end{itemize}

\chapter{Evaluation}
\begin{itemize}
  \item Functional tests (OAuth flow, HR retrieval, exercise recording).
  \item Performance (poll intervals, response times).
  \item Reliability under constraints (429 rate limits, no-sensor data scenarios).
  \item Usability/heuristic review (if no user study).
\end{itemize}

\chapter{Results}
\begin{itemize}
  \item Present findings from tests/experiments.
  \item Screenshots or figures of the dashboard and exercise runner.
  \item Any measured metrics (latency, success rate, correctness of HR fallback).
\end{itemize}

\chapter{Discussion}
\begin{itemize}
  \item Interpret results relative to objectives/questions.
  \item Limitations (Fitbit data availability, rate limits, pose estimation robustness).
  \item Threats to validity.
\end{itemize}

\chapter{Future Work}
\begin{itemize}
  \item Alternative wearables/providers; better buffering/caching.
  \item Improved coaching/pose feedback; clinician analytics; offline support.
\end{itemize}

\chapter{Conclusion}
\begin{itemize}
  \item Summarize contributions and outcomes.
  \item Reflect on impact and next steps.
\end{itemize}

\printbibliography

\appendix
\chapter{Appendix}
\begin{itemize}
  \item Environment setup and run instructions.
  \item API endpoints summary.
  \item Additional figures or code listings as needed.
\end{itemize}

\end{document}
